\chapter{Differential Kinematics and Statics}

\subsection{Differential Kinematics}
Given $\dot{q} \in \mathbb{R}^n$ and $v_e \in \mathbb{R}^r$, the mapping is defined by the Jacobian:
\[ v_e = J(q)\dot{q} \]

In redundant systems, certain joint velocities result in zero end-effector velocity (\textit{internal motions}).
\begin{itemize}
    \item \textbf{If Jacobian has full rank:} 
    \[ \dim(\mathcal{R}(J)) = r, \quad \dim(\mathcal{N}(J)) = n - r \]
    \item \textbf{Always true (Rank-Nullity Theorem):} 
    \[ \dim(\mathcal{R}(J)) + \dim(\mathcal{N}(J)) = n \]
\end{itemize}

\subsection{General Solution}
If $\dot{q}^*$ is a solution to $v_e = J\dot{q}$, we can choose a matrix $P \in \mathbb{R}^{n \times n}$ such that $\mathcal{R}(P) \equiv \mathcal{N}(J)$. The general solution is:
\[ \dot{q} = \dot{q}^* + P\dot{q}_0 \]
where $\dot{q}_0$ is an \textbf{arbitrary} vector. The matrix $P$ ensures that $\dot{q}_0$ produces only internal motions:
\[ J\dot{q} = J\dot{q}^* + \underbrace{JP\dot{q}_0}_{=0} = J\dot{q}^* = v_e \]
Since $P\dot{q}_0 \in \mathcal{N}(J)$, these motions do not affect the end-effector velocity.

\section{Redundant Case}

In the redundant case, the Jacobian matrix $J$ has dimensions $r \times n$ with $r < n$.
\[ v_e = J(q)\dot{q} \]
In this scenario, it is not possible to invert $J(q)$ directly. To find $\dot{q}$, a cost function must be minimized:
\[ g(\dot{q}) = \frac{1}{2} \dot{q}^T W \dot{q} \]
where $W$ is an $n \times n$ positive definite matrix ($W > 0 \Rightarrow g(\dot{q}) > 0$).

\subsection{Lagrangian Method}
The optimization problem is formulated as:
\[ \begin{cases} \min g(\dot{q}) = \frac{1}{2} \dot{q}^T W \dot{q} \\ \text{s.t. } v_e = J(q)\dot{q} \end{cases} \]
Using the Lagrangian:
\[ g(\dot{q}, \lambda) = \frac{1}{2} \dot{q}^T W \dot{q} + \lambda^T (v_e - J\dot{q}) \]
Setting the partial derivatives to zero:
\begin{enumerate}
    \item $\left( \frac{\partial g}{\partial \dot{q}} \right)^T = (\dot{q}^T W - \lambda^T J)^T = 0 \Rightarrow W^T \dot{q} - J^T \lambda = 0 \Rightarrow \dot{q} = W^{-1} J^T \lambda$
    \item $\left( \frac{\partial g}{\partial \lambda} \right)^T = (v_e - J\dot{q})^T = 0 \Rightarrow v_e^T - \dot{q}^T J^T = 0 \Rightarrow v_e = J\dot{q}$
\end{enumerate}

\subsection{Solution for Joint Velocities}
Substituting the expression for $\dot{q}$ into the constraint:
\[ v_e = J W^{-1} J^T \lambda \]
Assuming the Jacobian has \textbf{full rank} and we are not in a singularity, the dimensions are $[r \times n][n \times n][n \times r] = [r \times r]$.
Solving for $\lambda$:
\[ \lambda = (J W^{-1} J^T)^{-1} v_e \]
The optimal solution is:
\[ \dot{q} = W^{-1} J^T (J W^{-1} J^T)^{-1} v_e \]
This is the only solution among the infinite possible ones that minimizes $g(\dot{q})$.

\paragraph{Case $W = I$}
If we weight the joints equally ($W = I$), the solution simplifies to the \textbf{right pseudo-inverse} $J^+$:
\[ \dot{q} = J^T (J J^T)^{-1} v_e = J^+ v_e \quad \text{where } J J^+ = I \]

\subsection{Secondary Constraints and Internal Motions}
If redundancy exists, we can use an arbitrary vector $\dot{q}_0$ to satisfy secondary constraints. The new optimization problem becomes:
\[ \begin{cases} g(\dot{q}) = \frac{1}{2} (\dot{q} - \dot{q}_0)^T (\dot{q} - \dot{q}_0) \\ \text{s.t. } v_e = J\dot{q} \end{cases} \]
The Lagrangian leads to the solution:
\[ \dot{q} = \dot{q}_0 + J^T (J J^T)^{-1} (v_e - J\dot{q}_0) \]
Rewriting using $J^+$:
\[ \dot{q} = \dot{q}_0 + J^+ (v_e - J\dot{q}_0) = \dot{q}_0 + J^+ v_e - J^+ J \dot{q}_0 \]
\[ \dot{q} = J^+ v_e + (I_n - J^+ J) \dot{q}_0 \]
The term $P = (I_n - J^+ J)$ is a projection matrix. Verified by multiplying by $J$:
\[ J\dot{q} = \underbrace{J J^+}_{I} v_e + \underbrace{(J - J J^+ J)}_{0} \dot{q}_0 = v_e \]

\subsection{How to choose $\dot{q}_0$}
The vector $\dot{q}_0$ is typically chosen as the gradient of a secondary cost function $w(q)$:
\[ \dot{q}_0 = k_0 \left( \frac{\partial w(q)}{\partial q} \right)^T \]
Examples of $w(q)$:
\begin{itemize}
    \item \textbf{Manipulability:} $w(q) = \sqrt{\det(J(q) J^T(q))}$. Maximizing this ensures the Jacobian is far from singularity.
    \item \textbf{Obstacle Avoidance:} $w(q) = \min_{p, o} \| p(q) - o \|$ (distance from an obstacle).
\end{itemize}

\section{Redundant Case}

In the redundant case, the Jacobian matrix $J$ is $r \times n$ with $r < n$.
\[ v_e = J(q)\dot{q} \]
Since $J(q)$ cannot be inverted directly, we find $\dot{q}$ by minimizing a cost function:
\[ g(\dot{q}) = \frac{1}{2} \dot{q}^T W \dot{q} \quad \text{with } W_{n \times n} > 0 \Rightarrow g(\dot{q}) > 0 \]

\subsection{Lagrangian Method}
The optimization problem is:
\[ \begin{cases} \min g(\dot{q}) = \frac{1}{2} \dot{q}^T W \dot{q} \\ \text{sbj. } v_e = J(q)\dot{q} \end{cases} \]
Using the Lagrangian:
\[ g(\dot{q}, \lambda) = \frac{1}{2} \dot{q}^T W \dot{q} + \lambda^T (v_e - J\dot{q}) \]
Setting derivatives to zero:
\begin{enumerate}
    \item $\left( \frac{\partial g}{\partial \dot{q}} \right)^T = (\dot{q}^T W - \lambda^T J)^T = 0 \Rightarrow W^T \dot{q} - J^T \lambda = 0 \Rightarrow \dot{q} = W^{-1} J^T \lambda$
    \item $\left( \frac{\partial g}{\partial \lambda} \right)^T = (v_e - J\dot{q})^T = 0 \Rightarrow v_e = J\dot{q}$
\end{enumerate}

\subsection{Solution for Joint Velocities}
Substituting the previous results:
\[ v_e = J W^{-1} J^T \lambda \quad \text{(Assuming full rank, not in a singularity)} \]
\[ \lambda = (J W^{-1} J^T)^{-1} v_e \Rightarrow \dot{q} = W^{-1} J^T (J W^{-1} J^T)^{-1} v_e \]
If $W = I$ (equally weighted joints), we obtain the \textbf{right pseudo-inverse} $J^+$:
\[ \dot{q} = J^T (J J^T)^{-1} v_e = J^+ v_e \quad \text{with } J J^+ = I \]

\subsection{Secondary Constraints and Internal Motions}
If redundancy exists, $\dot{q}_0$ can satisfy secondary constraints:
\[ \dot{q} = J^+ v_e + (I_n - J^+ J) \dot{q}_0 \]
Where $P = (I_n - J^+ J)$ is the projection matrix into the null space of $J$.

\paragraph{How to choose $\dot{q}_0$}
$\dot{q}_0 = k_0 \left( \frac{\partial w(q)}{\partial q} \right)^T$ where $w(q)$ is a secondary cost function:
\begin{itemize}
    \item \textbf{Manipulability}: $w(q) = \sqrt{\det(J(q)J^T(q))}$ (far from singularity).
    \item \textbf{Obstacle Avoidance}: $w(q) = \min_{p,o} \| p(q) - o \|$.
\end{itemize}

\subsection{Singularity Handling (Non-Full Rank $J$)}
When $J$ is not full rank, two possibilities arise:
\begin{itemize}
    \item $v_e \in \mathcal{R}(J)$: In a singularity, but $v_e$ is \textbf{admissible}.
    \item $v_e \notin \mathcal{R}(J)$: Nothing we can do.
\end{itemize}
In the neighborhood of a singularity, a small $v_e$ leads to a very large $\dot{q}$ due to the small determinant.

\paragraph{Damped Least-Squares (DLS) Inverse}
The solution is the DLS inverse $J^*$:
\[ J^* = J^T (J J^T + k^2 I)^{-1} \]
This comes from minimizing the error $(v_e - J\dot{q})$ without imposing it to be zero, preventing $\dot{q}$ from becoming too big:
\[ g''(\dot{q}) = \frac{1}{2} (v_e - J\dot{q})^T (v_e - J\dot{q}) + \frac{1}{2} k^2 \dot{q}^T \dot{q} \]

\section{Singularity Handling}

\subsection{Problem Definition}
When the Jacobian $J$ is not full rank, we face two possibilities:
\begin{itemize}
    \item $v_e \in \mathcal{R}(J)$: We are in a singularity, but $v_e$ is \textbf{admissible}.
    \item $v_e \notin \mathcal{R}(J)$: Nothing we can do.
\end{itemize}
Even in the neighborhood of a singularity, a small $v_e$ results in a large $\dot{q}$ due to the small determinant.

\subsection{Damped Least-Squares (DLS) Inverse}
The solution to this problem is the DLS inverse $J^*$:
\[ J^* = J^T (J J^T + k^2 I)^{-1} \]
This is derived from a minimization problem where we minimize the error $(v_e - J\dot{q})$ without forcing it to zero, preventing joint velocities from becoming too large:
\[ g''(\dot{q}) = \frac{1}{2} (v_e - J\dot{q})^T (v_e - J\dot{q}) + \frac{1}{2} k^2 \dot{q}^T \dot{q} \]

\section{Analytical Jacobian}

\subsection*{Linear and Angular Velocity}
The linear velocity is the derivative of the position with respect to the joints:
\[ \dot{p}_e = \frac{\partial p}{\partial q} \dot{q} = J_p(q)\dot{q} \]

For the orientation:
\[ q \xrightarrow{\text{Direct}} R_e \xrightarrow{\text{Inv.}} \phi_e \]

\subsection{ZYZ Euler Angles Case}
$\dot{\phi} = \begin{bmatrix} \dot{\varphi} \\ \dot{\vartheta} \\ \dot{\psi} \end{bmatrix}$

Let's consider ZYZ:
\begin{enumerate}
    \item $\omega_z = \dot{\varphi} \begin{bmatrix} 0 \\ 0 \\ 1 \end{bmatrix}$
    \item Now the frame is rotated ($R_z$). Now I want to rotate about the $y'$ axis:
    \[ \omega^0 = \dot{\vartheta} R_z y' = \dot{\vartheta} \begin{bmatrix} c\varphi & -s\varphi & 0 \\ s\varphi & c\varphi & 0 \\ 0 & 0 & 1 \end{bmatrix} \begin{bmatrix} 0 \\ 1 \\ 0 \end{bmatrix} \Rightarrow \omega^0 = \dot{\vartheta} \begin{bmatrix} -s\varphi \\ c\varphi \\ 0 \end{bmatrix} \]
    $R_z y'$ is the $y'$ axis respect to the 0-frame.
    \item Now the rotation is $R_z R_y$:
    \[ R_{zy} = \begin{bmatrix} c\varphi c\vartheta & -s\varphi & c\varphi s\vartheta \\ s\varphi c\vartheta & c\varphi & s\varphi s\vartheta \\ -s\vartheta & 0 & c\vartheta \end{bmatrix} \]
    \[ \omega_0 = \dot{\psi} R_{zy} z'' = \dot{\psi} \begin{bmatrix} c\varphi s\vartheta \\ s\varphi s\vartheta \\ c\vartheta \end{bmatrix} \]
\end{enumerate}



The total angular velocity is:
\[ \omega_e = \begin{bmatrix} 0 & -s\varphi & c\varphi s\vartheta \\ 0 & c\varphi & s\varphi s\vartheta \\ 1 & 0 & c\vartheta \end{bmatrix} \dot{\phi} = T(\phi_e) \dot{\phi} \]
$\det T = -s^2\varphi s\vartheta - c^2\varphi s\vartheta = -(s^2\varphi + c^2\varphi) s\vartheta = -\sin\vartheta$
$\det T = 0$ if $\vartheta = 0, \pi \Rightarrow$ \textbf{REPRESENTATION SINGULARITY}

\subsection{Relationship between Jacobians}
\[ v_e = \begin{bmatrix} I & 0 \\ 0 & T(\phi_e) \end{bmatrix} \dot{x} = T_A(\phi_e) \dot{x} \quad \text{where } \dot{x} = \begin{bmatrix} \dot{p}_e \\ \dot{\phi}_e \end{bmatrix} \]
$J = T_A(\phi_e) J_A$
$J_A = T_A^{-1}(\phi_e) J$

$J_A$ is worse because it suffers of both the singularities of $T_A^{-1}(\phi_e)$ and $J$.
The Geometrical Jacobian is more "physical" ($w$ is the angular velocity).
In the Analytical Jacobian you use the derivative of parameters we use to describe the pose (Euler, Quaternion...).
$\dot{\phi}$ is the derivative of $\phi$ used to describe the orientation of the end-effector (not physical).

\textbf{SOMETIMES $J = J_A$}

\section{Inverse Kinematics Algorithms}

The basic numerical approach for inverse kinematics is:
\[ q(t_{k+1}) = q(t_k) + J^{-1}(q(t_k)) v_e(t_k) \Delta t \]
It is a \textbf{numerical integration} $\implies$ \textbf{DRIFT}.
This means that the end-effector pose corresponding to the computed joint variables differs from the desired one.
$\implies$ We need a \textbf{closed-loop control} that accounts for the operational space error.

\subsection{Closed-Loop Inverse Kinematics (CLIK)}
Defining the operational space error:
\[ e = x_d - x_e \]
The error dynamics are:
\[ \dot{e} = \dot{x}_d - \dot{x}_e = \dot{x}_d - J_A(q)\dot{q} \]
(Note: We used operational space variables $\implies$ \textbf{Analytical Jacobian}).

We have to find a relationship between $\dot{q}$ and $e$ in order to have a differential equation describing the error over time.
If $e \to 0 \implies 0 = \dot{x}_d - J_A(q)\dot{q} \implies \dot{q} = J_A^{-1}(q)\dot{x}_d$ (Open Loop).

\subsection{Jacobian (Pseudo) Inverse}
Assuming $J_A$ is square and non-singular, we choose:
\[ \dot{q} = J_A^{-1}(q) (\dot{x}_d + Ke) \]
where $K$ is a feedback term proportional to the error.
Replacing this in $\dot{e} = \dot{x}_d - J_A(q)\dot{q}$:
\[ \dot{e} = \dot{x}_d - J_A J_A^{-1} (\dot{x}_d + Ke) = -Ke \]
\[ \dot{e} + Ke = 0 \]
If $K > 0$ is a diagonal matrix $\implies$ the system is \textbf{asymptotically stable}:
\[ e(t) = e(0) e^{-Kt} \]

\paragraph{For a Redundant Manipulator:}
The solution incorporates the null-space projection:
\[ \dot{q} = J_A^{\dagger} (\dot{x}_d + Ke) + (I - J_A^{\dagger} J_A) \dot{q}_0 \]

\subsection{Control Scheme}

\begin{figure}[h]
    \centering
    \includegraphics[width=0.8\textwidth]{Capitolo 3/Immagini/control_scheme}
    \caption{Closed-loop inverse kinematics control scheme.}
\end{figure}

\begin{itemize}
    \item The larger the eigenvalues of $K$, the faster the error convergence.
    \item But since we use discrete-time systems, we cannot make it too large.
\end{itemize}

\section{Jacobian Transpose}

Instead of inverting the Jacobian, we can use the \textbf{Jacobian Transpose}. This approach is computationally lighter and avoids problems related to matrix inversion near singularities.

Consider the Lyapunov function candidate:
\[ V(e) = \frac{1}{2} e^T K^{-1} e > 0 \]
where $K$ is a positive definite symmetric matrix.
To guarantee asymptotic stability, we need $\dot{V}(e) < 0$.

Taking the derivative:
\[ \dot{V} = e^T K^{-1} \dot{e} = e^T K^{-1} (\dot{x}_d - J_A \dot{q}) \]
Assuming a constant desired pose ($\dot{x}_d = 0$):
\[ \dot{V} = -e^T K^{-1} J_A \dot{q} \]
If we choose:
\[ \dot{q} = J_A^T K e \]
Then:
\[ \dot{V} = -e^T K^{-1} J_A J_A^T K e = -(J_A^T e)^T (J_A^T e) \leq 0 \]
The error will converge to zero as long as the system is not in a singularity where $J_A^T e = 0$ with $e \neq 0$.

\subsection{Comparison: Inverse vs. Transpose}
\begin{itemize}
    \item \textbf{Inverse}: Provides a linear error dynamics ($\dot{e} + Ke = 0$), meaning the convergence rate is predictable and constant.
    \item \textbf{Transpose}: The convergence depends on the state of the manipulator (the Jacobian) and is generally slower, but it is much more robust.
\end{itemize}

\subsection{Transpose Control Scheme}

\begin{figure}[h]
    \centering
    \includegraphics[width=0.8\textwidth]{Capitolo 3/Immagini/transpose_scheme}
    \caption{Inverse kinematics control scheme using the Jacobian Transpose.}
\end{figure}



\section{Anthropomorphic Arm: Shoulder Singularity}

Let's consider an anthropomorphic arm in the shoulder singularity.

\begin{figure}[h]
    \centering
    \includegraphics[width=0.3\textwidth]{Capitolo 3/Immagini/anthropomorphic_shoulder_singularity}
    \caption{Anthropomorphic Shoulder Singularity.}
\end{figure}

\begin{equation}
p_e = \begin{bmatrix} 0 \\ 0 \\ p_z \end{bmatrix}
\end{equation}

\begin{equation}
dim(\mathcal{R}(J_p)) = 2
\end{equation}

I can only move in 2 dimensions.

\begin{equation}
\mathcal{R}(J_p) \equiv \mathcal{N}^{\perp}(J_p^T)
\end{equation}

\begin{equation}
J_p(q) = \begin{bmatrix} 0 & -c_1(a_2 s_2 + a_3 s_{23}) & -a_3 c_1 s_{23} \\ 0 & -s_1(a_2 s_2 + a_3 s_{23}) & -a_3 s_1 s_{23} \\ 0 & 0 & a_3 c_{23} \end{bmatrix}
\end{equation}

\begin{equation}
v_e = J_p \begin{pmatrix} \dot{\alpha} \\ \dot{\beta} \\ \dot{\gamma} \end{pmatrix}
\end{equation}

\section{Orientation Error}

\subsection{Position}

\begin{figure}[h]
    \centering
    \includegraphics[width=0.3\textwidth]{Capitolo 3/Immagini/orientation_error.png}
    \caption{}
\end{figure}

\begin{equation}
e_p = p_d - p_e(q)
\end{equation}
\begin{equation}
\dot{e}_p = \dot{p}_d - \dot{p}_e
\end{equation}

\subsection{Euler Angles}

\begin{equation}
e_o = \phi_d - \phi_e(q)
\end{equation}
\begin{equation}
\dot{e}_o = \dot{\phi}_d - \dot{\phi}_e(q)
\end{equation}

\begin{equation}
\dot{q} = J_A^{-1}(q) \begin{bmatrix} \dot{p}_d + K_p e_p \\ \dot{\phi}_d + K_o e_o \end{bmatrix}
\end{equation}

\section{Angle and Axis}

\begin{figure}[h]
    \centering
    \includegraphics[width=0.3\textwidth]{Capitolo 3/Immagini/angle_and_axis.png}
    \caption{}
\end{figure}

\begin{equation}
R(\theta, r) = R_d R_e^T
\end{equation}
Mutual rotation matrix (same origin).

\begin{equation}
-\frac{\pi}{2} < \theta < \frac{\pi}{2} \quad \text{IT'S OK BECAUSE THE ERROR IS SMALL}
\end{equation}
$e_o = 0$ if $\theta = 0, \theta = \pi$.

\begin{equation}
e_o = r \sin \theta = \frac{1}{2} (n_e \times n_d + s_e \times s_d + a_e \times a_d)
\end{equation}

\begin{equation}
\dot{e}_o = L^T \omega_d - L \omega_e
\end{equation}

\begin{equation}
L = -\frac{1}{2} (S(n_d)S(n_e) + S(s_d)S(s_e) + S(a_d)S(a_e))
\end{equation}

\begin{equation}
\Rightarrow \dot{e} = \begin{bmatrix} \dot{e}_p \\ \dot{e}_o \end{bmatrix} = \begin{bmatrix} \dot{p}_d - J_p(q) \dot{q} \\ L^T \omega_d - L J_o(q) \dot{q} \end{bmatrix} = \begin{bmatrix} \dot{p}_d \\ L^T \omega_d \end{bmatrix} - \begin{bmatrix} I & 0 \\ 0 & L \end{bmatrix} J \dot{q}
\end{equation}

WE USED THE GEOMETRIC JACOBIAN

\begin{equation}
\dot{q} = J^{-1}(q) \begin{bmatrix} \dot{p}_d + K_p e_p \\ L^{-1}(L^T \omega_d + K_o e_o) \end{bmatrix}
\end{equation}

\section{Unit Quaternions}

\begin{equation}
\Delta Q = Q_d \cdot Q_e^{-1} = \{ \Delta \eta, \Delta \epsilon \}
\end{equation}
I ONLY USE $\Delta \epsilon$

\begin{equation}
e_o = \Delta \epsilon = \eta_e(q) \epsilon_d - \eta_d \epsilon_e(q) - S(\epsilon_d) \epsilon_e(q)
\end{equation}

\begin{equation}
\dot{q} = J^{-1}(q) \begin{bmatrix} \dot{p}_d + K_p e_p \\ \omega_d + K_o e_o \end{bmatrix}
\end{equation}

\section{Second Order Algo.}

We know $\dot{x}_e = J_A(q) \dot{q}$ so:

\begin{equation}
\ddot{x}_e = J_A(q) \ddot{q} + \dot{J}_A(q, \dot{q}) \dot{q}
\end{equation}

\begin{equation}
\ddot{q} = J_A^{-1}(q) \left( \ddot{x}_d + K_D \dot{e} + K_P e - \dot{J}_A(q, \dot{q}) \dot{q} \right)
\end{equation}

\begin{equation}
\Rightarrow \ddot{e} + K_D \dot{e} + K_P e = 0 \quad K_D, K_P > 0 \Rightarrow e \rightarrow 0
\end{equation}

\section{Comparison}

\textit{3 links planar arm.}

\begin{equation}
x_e = h(q)
\end{equation}

\begin{equation}
\begin{bmatrix} p_x \\ p_y \\ \phi_e \end{bmatrix} = \begin{bmatrix} a_1 c_1 + a_2 c_{12} + a_3 c_{123} \\ a_1 s_1 + a_2 s_{12} + a_3 s_{123} \\ \theta_1 + \theta_2 + \theta_3 \end{bmatrix}
\end{equation}

\begin{equation}
J_A = \begin{bmatrix} -a_1 s_1 - a_2 s_{12} - a_3 s_{123} & -a_2 s_{12} - a_3 s_{123} & -a_3 s_{123} \\ a_1 c_1 + a_2 c_{12} + a_3 c_{123} & a_2 c_{12} + a_3 c_{123} & a_3 c_{123} \\ 1 & 1 & 1 \end{bmatrix}
\end{equation}

\begin{equation}
q_i = \begin{bmatrix} \pi & -\frac{\pi}{2} & -\frac{\pi}{2} \end{bmatrix}^T \quad \text{Initial configuration}
\end{equation}

\begin{equation}
p_d(t) = \begin{bmatrix} 0.25 (1 - \cos \pi t) \\ 0.25 (2 + \sin \pi t) \end{bmatrix}
\end{equation}
\begin{equation}
\phi_d = \sin \frac{\pi}{24} t \quad 0 \leq t \leq 4
\end{equation}
Desired trajectory

\section{On Matlab}

\begin{equation}
q(t_{n+1}) = q(t_n) + \dot{q}(t_n) \Delta t \quad \Delta t = 1ms \text{}
\end{equation}

Let's see with the simple inverse Jacobian

\begin{equation}
\dot{q} = J_A^{-1}(q) \dot{x}_d
\end{equation}

\begin{figure}[H]
    \centering
    \includegraphics[width=0.6\textwidth]{Capitolo 3/Immagini/drift_plots.png}
\end{figure}

$\Rightarrow$ Drift 

\section{Let's use the feedback error}

\begin{equation}
\dot{q} = J_A^{-1} (\dot{x}_d + K e) \quad K = diag \{500, 500, 100\} \text{}
\end{equation}

\begin{itemize}
    \item If we don't care about the orientation:
    \begin{equation}
    r = 2 \quad n = 3
    \end{equation}
    \begin{equation}
    \dot{q} = J_p^{\dagger} (\dot{p}_d + K_p e_p) \quad K_p = diag \{500, 500\}
    \end{equation}

    \item We can also use the transpose
    \begin{equation}
    \dot{q} = J_p^T(q) K_p e_p \quad K_p = diag \{500, 500\}
    \end{equation}
    $\Rightarrow$ Computationally lighter

    \item If it's redundant:
    \begin{equation}
    \dot{q} = J_p^{\dagger}(q) (\dot{p}_d + K_p e_p) + (I - J_p^{\dagger} J_p) \dot{q}_0
    \end{equation}
\end{itemize}

Manipulability measure:

\begin{equation}
w(\theta_1, \theta_2) = \frac{1}{2} (s_2^2 + s_3^2) \quad \text{Avoid singularity}
\end{equation}

\begin{equation}
\dot{q}_0 = k_0 \left( \frac{\partial w}{\partial q} \right)^T \quad k_0 = 50 \text{}
\end{equation}

\section{Statics}

Relationship between the forces on the end-effector and the forces on the joints (linear forces for prismatic joints and torques for revolute joints) with the manipulator at the equilibrium.

Elementary work done by torques:
\begin{equation}
dW_{\tau} = \tau^T dq \quad \tau \in \mathbb{R}^{n \times 1}, \gamma_e = \begin{bmatrix} f_e \\ \mu_e \end{bmatrix} \in \mathbb{R}^{r \times 1}
\end{equation}


Elementary work done by forces:
\begin{align}
dW_{\gamma} &= f_e^T dp_e + \mu_e^T \omega_e dt = \\
&= f_e^T J_p(q) dq + \mu_e^T J_o(q) dq = \\
&= \gamma_e^T J(q) dq \quad \text{Geometric Jacobian}
\end{align}


Note: $dp_e = J_p(q) dq$ and $\omega_e = J_o(q) \frac{dq}{dt}$.

\section{Virtual Works Principle}

Virtual work of the joints:
\begin{equation}
\delta W_{\tau} = \tau^T \delta q
\end{equation}


Virtual work of the end-effector:
\begin{equation}
\delta W_{\gamma} = \gamma_e^T \delta x = \gamma_e^T J(q) \delta q
\end{equation}


Static Equilibrium $\Rightarrow$ The sum of the virtual works must be zero.
\begin{equation}
\Rightarrow \delta W_{\tau} = \delta W_{\gamma} \quad \forall \delta q
\end{equation}


So:
\begin{equation}
\tau^T \delta q = \gamma_e^T J(q) \delta q \quad \forall q
\end{equation}


\begin{equation}
\Rightarrow \tau = J^T(q) \gamma_e
\end{equation}


\section{Kineto-Statics Duality}

\begin{figure}[H]
    \centering
    \includegraphics[width=0.7\textwidth]{Capitolo 3/Immagini/kineto_statics_mapping.png}
    \caption{Mapping between joint space and operational space for velocities and forces.}
\end{figure}

The relationship $v_e = J(q) \dot{q}$ maps joint velocities to end-effector velocities.
The relationship $\tau = J^T(q) \gamma_e$ maps end-effector forces to joint torques.

If $\mathcal{N}(J^T) \neq \emptyset$ means that you can find some forces $\gamma_e$ that don't require a $\tau$ in the joints to stay in equilibrium.

\subsection{Singularity Analysis}

\begin{figure}[H]
    \centering
    \includegraphics[width=0.2\textwidth]{Capitolo 3/Immagini/arm_singularity_statics.png}
\end{figure}

\begin{itemize}
    \item $\mathcal{R}(J) = \mathcal{N}^{\perp}(J^T)$
    \item $\mathcal{N}(J) = \mathcal{R}^{\perp}(J^T)$
\end{itemize}

$\gamma_e \in \mathcal{N}(J^T)$ is totally absorbed by the structure.

\section{Manipulability Ellipsoid}

$\dot{q}^T \dot{q} = 1$ is the set of unit joint velocities.
It describes the points on a surface of a sphere in the joint velocity space.
Now we want to see which velocities in the operational space are generated.

Knowing: $\dot{q} = J^{\dagger}(q) v_e$

\begin{equation}
v_e^T (J^{\dagger T} J^{\dagger}) v_e = 1
\end{equation}


Where $J^{\dagger} = J^T (J J^T)^{-1}$.

\begin{equation}
v_e^T \left( (J J^T)^{-1} J J^T (J J^T)^{-1} \right) v_e = 1
\end{equation}


\begin{equation}
v_e^T (J J^T)^{-1} v_e = 1
\end{equation}


(I don't use $\dot{q}_0$ because it doesn't affect the end effector).

\subsection{Properties of the Ellipsoid}

If I compute the eigenvectors of $J J^T$, I find the principal axis of the ellipsoid.
The dimensions are given by the singular values of $J$: $\sigma_i = \sqrt{\lambda_i (J J^T)}$ for $i=1, \dots, r$, where $\lambda_i (J J^T)$ is an eigenvalue of $J J^T$.

The volume of this ellipsoid is proportional to:
\begin{equation}
w(q) = \sqrt{\det(J J^T)}
\end{equation}


In the case of a non-redundant manipulator:
\begin{equation}
w(q) = |\det(J(q))|
\end{equation}


\section{Example: 2 Link Planar Arm}

\begin{equation}
w(q) = |\det(J)| = a_1 a_2 |s_2|
\end{equation}


Maximum for $\theta_2 = \pm \frac{\pi}{2}$.

\begin{figure}[H]
    \centering
    \includegraphics[width=0.3\textwidth]{Capitolo 3/Immagini/planar_ellipsoid_evolution.png}
    \caption{Evolution of the manipulability ellipsoid along the workspace.}
\end{figure}

Meaning: the last configuration is telling us that for a given $\dot{q}$ I will move profoundly in the vertical direction.
If it's fully stretched we lose a dimension.

\section{Force Ellipsoid}

We can do the same for the forces.

\begin{equation}
\tau^T \tau = 1 \Rightarrow \gamma_e^T (J(q) J^T(q)) \gamma_e = 1
\end{equation}


The eigenvectors are the same but the eigenvalues are the inverse.
A direction with a high velocity manipulability has a low force manipulability.

\begin{figure}[H]
    \centering
    \includegraphics[width=0.4\textwidth]{Capitolo 3/Immagini/velocity_force_duality.png}
    \caption{Duality between velocity (blue) and force (green) ellipsoids.}
\end{figure}
